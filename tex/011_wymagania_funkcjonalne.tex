\subsection{Wymagania funkcjonalne}
Głównym zadaniem systemu jest wsparcie zarządzania projektami pod kątem:

\begin{tabularx}{\textwidth}{|c|X|X|c|}
\hline 
Lp. & Nazwa & Opis & Priorytet \\ 
\hline 
\multicolumn{4}{|c|}{\textbf{Zarządzanie personelem}} \\
\hline 
F001 & Uwierzytelnianie & Aplikacja umożliwia uwierzytelnianie osób korzystających z systemu  & Wysoki \\ 
\hline 
F002 & Zarządzanie uprawnieniami & System umożliwia tworzenie kont z różnymi poziomami uprawnień & Wysoki \\ 
\hline 
F003 & Walidacja poziomu uprawnień & System dostosowuje widoczne dla użytkownika opcje w zależności od poziomu uprawnień użytkownika  & Wysoki \\ 
\hline 
F004 & Zarządzanie pracownikami & System umożliwia dodawanie usuwanie i modyfikację kont pracowników. W ramach konta jednego pracownika system umożliwia wprowadzenie: \newline
- danych personalnych \newline
- kwalifikacji \newline
- historii zatrudnienia \newline 
- stawki \newline
- informacji o stanowisku
& Wysoki \\
\hline 
F005 & Zarządzanie strukturą organizacyjną firmy & System umożliwia przeglądanie zależności stanowisk-owych pracowników firmy. System umożliwia tworzenie, usuwanie i modyfikacje elementów tak rozumianej struktury firmy. & Wysoki \\ 
\hline 
F006 & Zarządzanie czasem pracy pracowników & System umożliwia tworzenie raportów przez pracownika dotyczących przepracowanego czasu na poszczególnych zadaniach. W szczególności system umożliwia wprowadzenie informacji o stanie zaawansowania prac nad zadaniem którego dotyczy raport. & Wysoki \\ 
\hline 
\multicolumn{4}{|c|}{\textbf{Zarządzanie projektami}} \\
\hline 
F101 &  Tworzenie projektu & System umożliwia tworzenie i usuwanie projektów. System umożliwia wprowadzenie oraz modyfikację danych projektu np. status projektu. & Wysoki \\ 
\hline
F102 & Terminarz projektu & Dla każdego projektu system umożliwia tworzenie usuwanie i modyfikację terminarza projektu & Wysoki \\
\hline 
F103 & Definiowanie zadań & Dla każdego projektu system umożliwia dodanie i usunięcie zadania. System umożliwia także modyfikowanie danych zadania np. nazwa zadania, opis zadania & Wysoki \\ 
\hline 
F104 & Wymagania projektu & System umożliwia przypisanie do projektu kwalifikacji które musi posiadać zespół, aby można wykonać zaplanowaną pracę . & Wysoki \\ 
\hline 
F105 & Wyszukiwanie członków zespołów & System umożliwia wyszukanie pracowników o odpowiednich kwalifikacjach. Dla każdego znalezionego pracownika system pokazuje informacje o aktualnym obciążeniu pracownika& Wysoki \\ 
\hline 
F106 & Alokacja pracowników do projektów & System umożliwia przypisanie pracownika do projektu. & Wysoki \\ 
\hline 
F107 & Obciążenie pracowników & System udostępnia informacje o obciążeniu każdego z pracowników, czyli w nad jakimi projektami pracuje, oraz na jaki czas pracownik został 'zarezerwowany' w związku z danym projektem & Wysoki \\ 
\hline 
F108 & Zarządzanie kosztami projektów & System umożliwia: \newline
- definiowanie kosztów poszczególnych pracowników \newline
- zatwierdzanie raportów czasu pracy pracowników. & Wysoki \\
\hline
\multicolumn{4}{|c|}{\textbf{Monitorowanie statusu projektów}} \\
\hline 
F201 & Monitorowanie kosztów & System automatycznie informuje o przekroczeniu planowanych kosztów zadań lub całego projektu & Wysoki \\ 
\hline 
F202 & Zaawansowanie pracy & System automatycznie określa stan prac na podstawie danych podanych w raportach pracowników & Wysoki \\
\hline 
F203 & Podgląd zaawansowania pracy & System pozwala na podgląd zaawansowania prac nad całym projektem oraz w rozbiciu na poszczególne zadania & Wysoki \\
\hline 
\multicolumn{4}{|c|}{\textbf{Zarządzanie systemem}} \\
\hline 
F301 & Generowanie profili & System pozwala generować profile umiejętności kandydatów.  & Średni \\ 
\hline
F302 & Archiwizacja danych & System umożliwia archiwizacje danych związanych z projektami & Średni \\ 
\hline

\end{tabularx} 

