\section{Analiza wymagań}
System skierowany jest do firm zajmujących się wytwarzaniem produktów IT. Skierowany jest do wszystkich zatrudnionych w firmie, z głównym wskazaniem na:
\begin{itemize*}
\item menadżerów projektów,
\item pracowników technicznych (programistów, testerów)
\item analityków.
\end{itemize*}

System ma za zadanie wspomagać zarządzanie projektami pod kątem zasobów niezbędnych do ich wytwarzania. Ma umożliwiać stałe monitorowanie kosztów projektu i zapewniać możliwość szybkich reakcji w dynamicznie zmieniającym się środowisku.
\subsection{Wymagania funkcjonalne}
Głównym zadaniem systemu jest wsparcie zarządzania projektami pod kątem:
\begin{enumerate}
\item planowania i śledzenia czasu pracy pracowników firmy,
\item definiowania i raportowania kosztów projektów,
\item zarządzania dostępnym czasem pracowników,
\item przechowywaniem profili pracowników i możliwość ich wygodnego przeglądania i przeszukiwania,
\item definiowania struktury firmy (menadżerowie i podwładni),
\item monitorowania postępów projektów.
\end{enumerate}

Aplikacja będzie używana przez różne osoby w firmie. W ramach aplikacji zdefiniowane są następujące role, jakie pełnione są przez użytkowników:
\begin{itemize*}
\item \textbf{Menadżer zespołu} - w ramach tej roli ...
\item Menadżer projektu
\item Pracownik HR
\item Pracownik techniczny
\end{itemize*}
W ramach każdej roli zdefiniowany jest inny zakres możliwości i funkcji dostępnych w systemie.

